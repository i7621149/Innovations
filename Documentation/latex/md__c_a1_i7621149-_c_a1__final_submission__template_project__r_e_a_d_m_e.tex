\#\-Template Project

This is a template project, in a similar vein to ngl's Blank\-N\-G\-L. It contains the recommended starting point for any projects that use the shader library.

This template includes \hyperlink{class_shader_lib_pro}{Shader\-Lib\-Pro}, \hyperlink{struct_shader_variables}{Shader\-Variables}, \hyperlink{class_shader_set}{Shader\-Set}, \hyperlink{struct_shader_pro}{Shader\-Pro} and \hyperlink{class_entity}{Entity}, which are all required to use the library and are considered the components of the library.

It also includes an \hyperlink{class_n_g_l_scene}{N\-G\-L\-Scene} class and \hyperlink{class_background}{Background}, which inherits from \hyperlink{class_entity}{Entity}. Both of these are recommended for use with my classes, since they are good examples of how to use the library, and the \hyperlink{class_n_g_l_scene}{N\-G\-L\-Scene} will deal with the required \hyperlink{struct_shader_variables}{Shader\-Variables} automatically, so that the user does not need to set those up themselves. The \hyperlink{class_background}{Background} class is also recommended since it shows how to inherit from \hyperlink{class_entity}{Entity}, and means that there will always be something in the background rather than the clear color.

The user could rewrite or get rid of either of these if they prefer to use the library in their own way.

The template also comes with a default shader that shows how it can be written, in this case it is the same as the \char`\"{}new\char`\"{} option in Shader\-Toy.

\subsection*{shader text files}

The shader text files take the following format\-: ``` \#comments are made with \#s

\#start with S\-H\-A\-D\-E\-R and shader name. one M\-A\-I\-N is required. S\-H\-A\-D\-E\-R M\-A\-I\-N

\#specify fragment shader glsl file. This will be combined with sampler2\-Ds if there are textures and the Base\-Fragment.\-glsl file F\-R\-A\-G\-M\-E\-N\-T shaders/\-Default\-Quad\-Fragment.\-glsl

\#specify vertex shader glsl file. This will be combined with the Base\-Vertex.\-glsl V\-E\-R\-T\-E\-X shaders/\-Geo\-Vertex.\-glsl

\#textures can be specified with T\-E\-X\-T\-U\-R\-E followed by type (2\-D) and then the file T\-E\-X\-T\-U\-R\-E 2\-D textures/example.\-png T\-E\-X\-T\-U\-R\-E 2\-D textures/example2.\-png

\#end the shader E\-N\-D M\-A\-I\-N

``` 